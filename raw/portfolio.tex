\documentclass{portfolio}
\graphicspath{{portfolioImages/}}
\begin{document}

  \vfillForced

  \PortfolioTitle{Lindsay William Bradford}
 
  \vfillForced
  
  \begin{ContactDetails}
    \ContactDetail{Mobile Phone}{+61 423 989 511}
    \ContactDetail{Email}{\href{mailto:lindsay.w.bradford@gmail.com}{lindsay.w.bradford@gmail.com}}
  \end{ContactDetails}
  
  \vfillForced

  \tableofcontents 

  \vfillForced
  
  \begin{Projects}
    \begin{Project}{YAWLEditor}
      \ProjectTools{\tool{Java} \tool{Swing} \tool{JGraph} \tool{XML} \tool{XMLSchema} \tool{SAX}} 
      \ProjectTimeline{Sept 2003 -- Mar 2008}
      \ProjectContribution{Sole Designer and Principal Developer, up to and including release 2.0}
      \begin{ProjectOverview}
        YAWL is an open source Workflow/BPM system based on a concise and powerful modelling language jointly developed by QUT's BPM research group in 
        collaboration with Eindhoven University of Technology.  YAWL builds on the insights gained from their research into workflow patterns and 
        combines it with the powerful concurrency language of Petri-nets.  

        I was the principal author of the YAWLEditor (a graphing tool, capable of defining valid specifications to run within the YAWL engine) from its inception to 
        release 2.0 of the YAWLEditor. 
      \end{ProjectOverview}
      \ProjectImage{width=0.8\textwidth}{YAWLEditor001}
      \ProjectFurtherInfo{Source-code is available via \href{https://sourceforge.net/projects/yawl/}{Sourceforge}.\\A technical manual, discussing the design of the YALEditor is available upon request.}
    \end{Project}
    \begin{Project}{PersonalFinancier}
      \ProjectTools{\tool{Java} \tool{Swing} \tool{Gradle} \tool{JUnit} \tool{Mockito} \tool{json-io} \tool{Legion of the Bouncy Castle} \tool{GRAL}} 
      \ProjectTimeline{April 2014 -- Current}
      \ProjectContribution{Sole Developer}
      \begin{ProjectOverview}
        A budgetting and financial forecasting tool built to suit my way of managing household finances.
        Supports cash-flow analysis across accounts and budget categories, an inflation calculator to compare
        the changing value of money over time, full undo/redo behaviour coverage and budget file encryption via AES-256. 

        The final deployable artifact is a single file, fusing the core behaviour with its support libraries via an innovative usage of 
        Gradle. The 3rd-party encryption library is an exception to this rule, and may be optionally placed in the same directory as the 
        base artifact to enable file encryption behaviour.
      \end{ProjectOverview}
      \ProjectImage{width=\textwidth}{PersonalFinancier001}
      \ProjectFurtherInfo{Source-code is available via \href{https://github.com/LindsayBradford/PersonalFinancier}{GitHub}.}
    \end{Project}
    \begin{Project}{Various Personal Java Desktop Utilities}
      \ProjectTools{\tool{Java} \tool{Swing}}
      \ProjectTimeline{Ongoing}
      \ProjectContribution{Sole Developer.}
      \begin{ProjectOverview}
        Given that Java and Swing produce a rich desktop GUI interface, I've developed a number of personal tools in my spare time over the years to allow me to 
        explore various aspects of the language whilst delivering utility regardless of my operating system. Below is a list of the utilities constructed:
        \begin{itemize}
           \item The Compounder - A utility for calculating compound interest, and comparing alternate compounding scenarios.
           \item CPI Calculator - For comparing how the value of money changes over time, functionality folded into my PersonalFinancier tool.
           \item Share Trading Calculator - For calculating finacial risk, and benchmarking returns against cash. 
           \item Dividend Calculator - For calculating divident returns on Australian shares.
           \item Stopwatch - An excuse to lean about Java's threading model, and a convenient desktop timer.
           \item Shadowrun Dice Roller - For a roleplaying game needing a large number of dice.
        \end{itemize}
      \end{ProjectOverview}
      \ProjectImage{width=\textwidth}{JavaUtilities}
      \ProjectFurtherInfo{Demonstrations and/or browsing of source is available upon request.}
    \end{Project}
    \begin{Project}{Murrumbidgee Wetlands Relational Database Tools (MWRDTools)}
      \ProjectTools{\tool{C\#} \tool{ESRI ArcMap} \tool{ESRI ArcCatalog} \tool{SQLServer Lite}}
      \ProjectTimeline{September 2013 -- February 2014}
      \ProjectContribution{Project Manager, Sole Designer and Developer.}
      \begin{ProjectOverview}
        MWRDTools is an Esri `ArcMap for Desktop' addon that supplies a suite of decision-support functions for the Murrumbidgee Wetlands Relational Database (MWRD).
        It is a complete re-design and re-write of a tool commissioned for the Murrumbidgee Catchment Management Authority, allowing it to migrate to alternate user-interface
        technologies, and to replace the existing single-user spatial database with alternatives that support multi-user behaviour.
      \end{ProjectOverview}
      \ProjectImage{width=0.8\textwidth}{MWRDTools001}
      \ProjectImage{width=0.55\textwidth}{MWRDTools002}
      \ProjectFurtherInfo{Source-code is available via \href{https://github.com/LindsayBradford/MWRDTools}{GitHub}.}
    \end{Project}
    \begin{Project}{Murrumbidgee Wetland Condition Indicator Tool (MWCIT)}
      \ProjectTools{\tool{VB.NET} \tool{InnoScriptStudio} \tool{NUnit} \tool{NSubstitute}}
      \ProjectTimeline{September 2013 -- February 2014}
      \ProjectContribution{Project Manager, Sole Designer and Developer.}
      \begin{ProjectOverview}
        MWCIT s a tool commissioned by the Murrumbidgge Catchment Management Authority, allowing the user-configurable loose integration
        of a number of natural resource management tools and datasets into a single desktop launcher application. 
        The application has a layered architecture, implements the Model-View-Presenter (MVP) design pattern, and deploys as a stand-alone 
        executable thanks to the innovative embedding of support assemblies within the bootstrap assembly. 
      \end{ProjectOverview}
      \ProjectImage{width=0.4\textwidth}{MWCIT001}
      \ProjectImage{width=\textwidth}{MWCIT002}
      \ProjectFurtherInfo{Source-code is available via \href{https://github.com/LindsayBradford/MWCIT}{GitHub}.}
    \end{Project}
    \begin{Project}{Murray Darling Basin Authority Environmental Prioritisation Tool (MDBEAP)}
      \ProjectTools{\tool{Object-Pascal} \tool{C} \tool{InnoSetupStudio} \tool{Marxan} \tool{Zonae Cogito} \tool{MapWinGIS}}
      \ProjectTimeline{April 2015 -- November 2015}
      \ProjectContribution{Sole Spatial Database Designer and Developer.}
      \begin{ProjectOverview}
        MDBEAP s a tool commissioned as part of several deliverables to the Murray Darling Basin Authority (MDBA).  It, and it's companion tool MarxanHO
        represent a modification of the popular environmental prioritisation tool Marxan, and it's Desktop interface, Zonae Cogito to allow for a 
        prioritisation approach to conservation planning adapted to suit MDBA desires.  
      \end{ProjectOverview}
      \ProjectImage{width=\textwidth}{MDBEAP001}
      \ProjectFurtherInfo{MDEAP and MarxanHO source-code available via \href{https://github.com/LindsayBradford/MDBEAP}{GitHub}.}
    \end{Project}
  \end{Projects}
  \vfillForced
\end{document}
